% !TeX encoding=utf8
% !TeX spellcheck = en-US
% !TeX root=../UID_Project_Documentation.tex

%% Base packages

\usepackage{calc}	        % Calculation

\usepackage[                % Multi language support
    main=ngerman,
%    british,                % For the european date format in the bibliographies
%    ngerman,
]{babel}

\usepackage[utf8]{inputenc} % UTF-8 text encoding

\usepackage[		        % Color support with color mixing models
    dvipsnames, % Load a set of predefined colors
    table,      % Load the colortbl package
    % fixpdftex,  % Load the pdfcolmk package (may be problematic)
    hyperref,   % Support  the  hyperref  package
    fixinclude, % Prevent dvips color reset before .eps file inclusion
]{xcolor}

\usepackage[		        % Support for graphics in LaTeX
    %final,
%    draft 		% do not include images (faster)
]{graphicx}

\usepackage{wrapfig} % for wrapping text around figures

\usepackage{epstopdf}   % If an eps image is detected, epstopdf is
                        % automatically called to convert it to pdf format.

% \usepackage{ragged2e}	% environments for setting ragged text
						% which allow hyphenation.

%% Bugfixes

%\usepackage{marginnote} % marginnote allows a margin note, where \marginpar
%                        % fails
%
\usepackage{scrhack}    % Redefines implementations of
%                        % packages float, hyperref and listings
%
%\usepackage{marginfix}  % changes the \marginpar commands, such
%                        % that long margin notes work.
%
%\RequirePackage{xspace} %  Used to define commands that don't eat spaces.
%
%\usepackage[xspace]{ellipsis}   % fixes bug in ellipsis (...)

%% Fonts

\usepackage{relsize}    % Set the font size relative to the current font size

%% Math

\usepackage[            % basic math package
    centertags, % (default) center tags vertically
    %tbtags,    % 'Top-or-bottom tags': For a split equation, place equation
    % numbers level with the last (resp. first) line, if numbers
    % are on the right (resp. left).
    sumlimits,  %(default) Place the subscripts and superscripts of summation
    % symbols above and below
    %nosumlimits, % Always place the subscripts and superscripts of
    % summation-type symbols to the side, even in displayed
    % equations.
    intlimits,  % Like sumlimits, but for integral symbols.
    %nointlimits, % (default) Opposite of intlimits.
    namelimits, % (default) Like sumlimits, but for certain 'operator names'
    % such as det, inf, lim, max, min, that traditionally have
    % subscripts placed underneath when they occur in a displayed
    % equation.
    %nonamelimits, % Opposite of namelimits.
    %leqno,     % Place equation numbers on the left.
    %reqno,     % Place equation numbers on the right.
    fleqn,      % Position equations at a fixed indent from the left margin
    % rather than centered in the text column.
]{amsmath}

%\usepackage[fixamsmath,disallowspaces]{mathtools}   % The mathtools package is
%                                                    % an extension package to
%                                                    % amsmath. Furthermore it
%                                                    % corrects various bugs
%
%\usepackage[            % Inhibits the usage of plain TeX and of standard
%%LaTeX
%                        % math environments
%    all,
%    % warning
%    error
%]{onlyamsmath}
%
%\usepackage{braket}     % Macros for Dirac bra-ket notation and sets.
%
%\usepackage{cancel}     % strike out arguments in math mode
%
%\usepackage{empheq}     % Emphasize equations
%
%\usepackage{exscale}    % scales math mode output in all environments correct
%
%\usepackage{fixmath}    % fixes for the default Computer Modern math fonts
%
%\usepackage{icomma}     % Enables the correct use of the comma as a decimal
%                        % separator in math mode
%
%\usepackage{xfrac}      % LaTeX 3 Package for nice inline fractions
%                        % Provides: \sfrac{1}{2}
%
%%% Science
%
%\usepackage{siunitx}    % siunitx aims to provide a unified method to
%                        % typeset numbers and units correctly and easily.

%% Symbols

\usepackage[gen]{eurosym}   % European currency symbol

\usepackage{pifont}         % Common symbols

%% Tables

%\usepackage{booktabs}           % some additional commands to enhance the
%                                % quality of tables
%
%\usepackage{multirow, bigstrut} % extends the standard tabular environment
%%with
%                                % cells spanning over multiple rows.
%
%
%\usepackage{ltxtable}           % Table spanning over many pages (from
%                                % longtable package) and with strechable
%                                % columns (from tabularx package)
%
\usepackage{tabu}       % defines a single environment tabu to make all kinds
%                        % of tabulars. It is more flexible than tabular,
%                        % tabular*, tabularx and array and extends the
%                        % possibilities.
%
%\usepackage{tablestyles}

%% Text

% - Text decoration

\usepackage[normalem]{ulem} % commands for underlining for emphasis

\usepackage{soulutf8}       % commands for for emphasis

\usepackage{url}            % enable linebreaks for URLs

\usepackage{setspace}       % for custom line sapcing

\usepackage{float}          % better control over float environments

\usepackage[
    activate={true, compatibility},
    final,tracking=true,
%    kerning=true,
%    spacing=true,
    factor=1100,
    stretch=10
    ,shrink=10
]{microtype}

\microtypecontext{spacing=nonfrench}

% activate={true,nocompatibility} - activate protrusion and expansion
% final - enable microtype; use "draft" to disable
% tracking=true, kerning=true, spacing=true - activate these techniques
% factor=1100 - add 10% to the protrusion amount (default is 1000)
% stretch=10, shrink=10 - reduce stretchability/shrinkability (default is 20/20)

% - Footnotes

\usepackage[                % The footmisc package provides several different
%                            % customisations of the way foonotes are
%                            % represented. Fixes a LaTeX bug with option
%                            % 'bottom'
%    bottom,      % Footnotes appear always on bottom. This is necessary
%                 % especially when floats are used
%    stable,      % Make footnotes stable in section titles
    perpage,     % Reset on each page
%    %para,       % Place footnotes side by side of in one paragraph.
%    %side,       % Place footnotes in the margin
%    ragged,      % Use RaggedRight
%    %norule,     % suppress rule above footnotes
    multiple,    % rearrange multiple footnotes intelligent in the text.
%    %symbol,     % use symbols instead of numbers
]{footmisc}

% - Code listings

\usepackage{listings}

% - Quotes

\usepackage[
    autostyle,
    % german=guillemets
    ]{csquotes}

%% Bibiliography
\usepackage[
    backend=biber,
%    style=alphabetic,
    style=authoryear,
    maxcitenames=2,
%    citestyle=authortitle,
%    sorting=none
]{biblatex}

% - Abbreviations and Acronyms
\usepackage[
    acronym,
    toc
]{glossaries}
%\usepackage[
%    printonlyused,
%    withpage
%]{acronym}

% - Table of contents
\usepackage[nottoc]{tocbibind}

% - Manipulating Counters
\usepackage{chngcntr}

% - Prevent floats from going over sections
\usepackage[section]{placeins}

% - better titles
% \usepackage{titlesec}
% \newcommand{\sectionbreak}{\clearpage}
