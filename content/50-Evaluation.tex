% !TeX encoding=utf8
% !TeX spellcheck=de-DE
% !TeX root=../UID_Project_Documentation.tex

\section{Evaluation: User Tests}

\subsection{Ziele}

Ziel unseres User Tests war in erster Linie festzustellen, ob unser USP, also ein granulares, vollständig individuell zusammenstellbares Nachrichtenangebot, instantan erkannt wird.

Zudem wollten wir feststellen, ob das Prinzip der Feeds grundsätzlich verstanden wird und deren Umsetzung intuitiv für den Nutzer ist. Dies war uns sehr wichtig, da nur mit Feeds das Potenzial der App zum Tragen kommt. Ist deren Sinn unklar oder die Erstellung zu kompliziert, erlischt einer unserer größten Vorteile im Vergleich zu Konkurrenzprodukten.

Ein sehr generelles Ziel war aber natürlich auch Probleme bei der Nutzung herauszufinden und somit Schwachstellen in der Bedienbarkeit zu beseitigen.

\subsection{Testprobanden}

Unsere Testprobanden setzten sich zusammen aus fünf Studenten im Alter von etwa 20 Jahren, die unterschiedliche Erfahrungsstände in Bezug auf Nachrichtenapps haben.

Proband 1 ist weiblich, absolviert ein Studium mit Schwerpunkt Medien und pendelt, zur Zeit aber mit dem Auto, jedes Wochenende 70 km. Es ist aber denkbar, dass sie in naher Zukunft den Zug nutzen wird, sodass sie dort dann eventuell Nachrichten konsumieren möchte. Proband 1 hat bisher allerdings keine Erfahrung mit News-Apps.

Proband 2 und 3 sind ebenfalls weiblich und studieren dual an der DHBW Karlsruhe. Sie haben einen starken Bezug zu Technik und Smartphones. Auch Proband 5, allerdings männlich, interessiert sich sehr stark für neue Technologien und Entwicklungen, da er ein Studium im Bereich Informatik absolviert.

Proband 4 studiert Architektur, hat aber im Vergleich zu allen anderen Probanden keine große Affinität zu Technik, weshalb dieser Kandidat besonders wichtig für unseren User Test war.

Lediglich Proband 3 nutzt bisher eine App, um sich über aktuelle Geschehnisse und Nachrichten zu informieren.

Die meisten unserer Probanden entsprechen stark der Persona von Sabine, da sie einerseits technikaffin sind, andererseits aber auch ein begrenztes Einkommen haben. Zudem sind sie sehr gebildet und aufgrund ihres Studiums ist Zeit wohl das am meisten knappe Gut.

Leider konnten wir keine Person mittleren Alters zu einem User Test bewegen. Wir sind aber der Meinung, dass wir eine ausreichend unerfahrene und technik-averse Person einbringen konnten (Proband 4), die uns ähnliche Resultate wie eine an der Persona \enquote{Uli} ausgerichtete Person liefern dürfte.

\subsection{Gestellte Fragen und Aufgaben}

Um festzustellen, ob unsere Navigationselemente in Form von Tabs aussagekräftig genug sind und dem User von Anfang an klar machen, was die Besonderheit unserer App ist, fragten wir noch vor allen anderen Aufgaben und Fragen, welche Funktionalitäten bei Betrachten der Action Bar erwartet von ihnen erwartet werden.

Die erste Aufgabe bestand dann darin, einen Feed zu erstellen. Diese Aufgabe zielte einerseits darauf ab, zu erkennen, ob Feeds generell verstanden werden, andererseits aber auch darauf, ob ihre Erstellung einleuchtend ist.

Weitere Aufgaben und Fragen sollten dann aufzeigen, ob es sonstige Probleme mit der Bedienung , sowie Navigation gibt und uns noch einmal zeigen, ob der USP tatsächlich vollständig verstanden wurde.

Der vollständige Moderationsleitfaden mit allen Aufgaben und Fragen, sowie der Moderation an sich, ist im Anhang dieser Projektdokumentation zu finden.

\subsection{Ergebnis}

Der User Test hat gezeigt, dass unser USP im großen und Ganzen von den Usern enorm schnell erkannt wird. Ohne sich mit der App befasst zu haben, wurde von den meisten Testprobanden anhand der Tabs erkannt, dass das Nachrichtenangebot individualisierbar ist. Dies war im positiven Sinne überraschend.

Es wurde zudem recht schnell klar, dass Android Nutzer mit dem Design sehr schnell vertraut sind und instinktiv zu den geforderten beziehungsweise gewünschten Screens navigieren.

Vereinzelt gab es aber Probleme mit der Auswahl der Kategorien einzelner Zeitungen beim Erstellen bzw. Bearbeiten eines Feeds. Die Auswahlmöglichkeit wurde hier oftmals entweder übersehen oder nicht verstanden.

Beim Abonnieren von Inhalten im Explore-Modus wurde zudem erwartet, dass der dafür zuständige Button nicht am Artikelende, sondern im Kopfbereich des Screens positioniert ist.

Es wurde außerdem klar, dass die Feeds, welche standardmäßig erstellt sind, darunter zum Beispiel \enquote{Frühstück}, dazu führen, dass der User möglicherweise den Sinn der Feeds nicht begreift. Argumentiert wurde zum Beispiel, dass man morgens keine anderen Interessen hat, als beispielsweise abends, sodass ein Feed explizit für den Morgen sinnfrei ist.

Da die Zurück-navigieren-Funktion beim ersten Prototypen noch nicht vollständig implementiert war, entstand auch hierbei Verwirrung.

\subsection{Abgeleitete Verbesserungsmaßnahmen}

Aus den entstandenen Problemen und Schwierigkeiten wurden die folgenden Verbesserungsvorschläge abgeleitet.

Zunächst möchten wir natürlich die Problematik betreffend die Auswahl der Kategorien beim Erstellen von Feeds lösen. Dies geschieht durch eine verbesserte Menüführung, die nach einem neuen Prinzip funktioniert. Als ersten Auswahlschritt sieht der Benutzer alle Nachrichtenquellen und hat die Möglichkeit nach diesen zu suchen oder auch nach Kategorien zu filtern. Im zweiten Schritt kann der Benutzer dann Kategorien hinzufügen. Hat der Benutzer eine Kategorie aus einer Nachrichtenquelle selektiert, so sieht er einen Haken neben dem entsprechenden Titel auf der Übersichtsseite, um direkt zu erkennen, dass eine Kategorie in diesem Untermenü ausgewählt wurde. Um den neuen Feed zu speichern, wird ein Disketten-Symbol an dem für Android typischen Platz eingeführt.

Weiterhin werden wir die standardmäßig, vordefinierten Feeds durch aussagekräftigere, sowie sinnhaftere ersetzen.

Den Button, der dafür zuständig ist, Inhalte aus dem Explore-Modus heraus zu abonnieren, soll nun auch im Kopfbereich eines Artikels angezeigt werden, sodass er leichter gefunden werden kann. Dies sollte analog zu den Änderungen in anderen Ansichten durch ein neues Piktogramm in der Menüleiste umgesetzt werden.

Natürlich werden wir auch die Navigation, insbesondere das Zurück-navigieren verbessern, respektive vollständig implementieren.
Zudem wurden uns noch einige interessante Impulse bezüglich Funktionen der App gegeben. So sollte es eine Möglichkeit geben, nach Ländern oder Regionen zu filtern, sowie den Homescreen zu personalisieren, in dem man entscheidet, welche Artikel angezeigt werden.

Umsetzen möchten wir aber noch die Filtermöglichkeit nach Regionen beziehungsweise Ländern, da dies eine von uns bisher versäumte, aber sehr grundlegende Funktionalität darstellt.

Mit diesem Schritt im UCD-Prozess ist nun ein gesamter Zyklus durchlaufen worden, sodass im folgenden Kapitel die erste Iteration erfolgen kann.
