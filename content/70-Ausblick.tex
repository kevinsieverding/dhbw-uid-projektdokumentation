% !TeX encoding=utf8
% !TeX spellcheck=de-DE
% !TeX root=../UID_Project_Documentation.tex

\section{Ausblick}

Mit der Erstellung eines zweiten Prototypen befinden wir uns nun in der zweiten Iteration des Projektes bei der Prototyping-Phase. Nun müsste zumindest noch einmal ein User Test erfolgen, um entweder festzustellen, dass die App ausreichend ausgereift ist, um sie zu veröffentlichen, oder dass noch weitere Verbesserungsmaßnahmen abgeleitet werden müssen, sodass noch ein oder gar mehrere Iterationszyklen durchlaufen werden sollten.

Ist die App dann ausreichend ausgereift designed, hängen die nächsten Schritte davon ab, welches Vorgehensmodell gewählt wurde. Beim Wasserfallmodell wäre nun die Implementation der App, also die tatsächliche Umsetzung, an der Reihe. Wurde eine agile Vorgehensweise gewählt, wäre die tatsächliche Entwicklung der App hingegen parallel zum Designen der UI der App verlaufen, sodass man nun schon ein releasebares Produkt hätte.

Wäre unsere Gruppe daran interessiert, die App wirklich umzusetzen und zu veröffentlichen, wäre es uns auf jeden Fall möglich gewesen, an die aus dieser Projektarbeit gewonnen Artefakte, Erkenntnisse, wie auch Ideen anzuknüpfen, um so mit einem großen Vorsprung in die \enquote{reale Welt} zu starten. Allerdings fürchten wir, dass die Konkurrenz durch große Anbieter viel zu groß ist, sodass es sehr schwer wäre, sich am Markt zu etablieren. Zudem ist auch der Aufwandsfaktor bei der Gründung eines Unternehmens enorm groß, sodass sich dies nur schwer mit dem Studium verbinden ließe. Als letzter Grund wäre noch zu nennen, dass wir alle bereits bei Unternehmen angestellt sind, die hochgradige Zukunftsperspektiven versprechen, sodass wir uns dazu entschieden haben, uns lieber auf eine Karriere in diesen Unternehmen zu konzentrieren, als im jungen Alter bereits selbst Unternehmensgründer zu werden.
