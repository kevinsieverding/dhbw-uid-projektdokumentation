% !TeX encoding=utf8
% !TeX spellcheck=de-DE
% !TeX root=../UID_Project_Documentation.tex

\section{Motivation}

\subsection{Einordnung und Problemstellung}

Das von uns erdachte Produkt beschäftigt sich mit Problemen, die beim Anbieten und Konsumieren von ehemaligen Print-Inhalten im digitalen Zeitalter entstehen und die wir als von den bisherigen Angeboten ungelöst sehen.

Mit dem Einzug von digitalen Gerätschaften und dem Internet in die Haushalte und den Alltag eines großen Teils der Weltbevölkerung ist die Nachfrage nach digitalen Inhalten jedweder Form stark und stetig gestiegen. Auch der Journalismus ist von dieser Entwicklung nicht verschont geblieben und es wird seitdem versucht sich entsprechend anzupassen. Zum Beispiel war bereits am 25. Oktober 1994 mit \enquote{Spiegel Online} eine von Deutschlands bekanntesten Quellen für Journalismus über das Netz erreichbar. Heute vertreibt praktisch jede professionelle Publikation ihre Inhalte zumindest teilweise über einen Webauftritt und muss sich dabei nicht nur gegenüber der traditionellen Konkurrenz sondern auch einer schier unzählige Masse an privaten und halb-professionellen \enquote{Bloggern} behaupten.

Diese neuen Gefilde zeigen einige Dynamiken und Aspekte auf, die sich von denen der traditionellen Vertriebswelt für Journalismus stark unterscheiden. Der moderne Nutzer ist es gewohnt alle Inhalte die er wünscht direkt abrufbar zu haben. Es ist normal verschiedenste Quellen zu konsumieren, um entweder ihre Inhalte zu einem bestimmten Thema zu vergleichen oder vielmehr sie den Schwerpunkten ihrer Kompetenz gerecht zu nutzen um sich so eine persönliche, optimierte Konsumlandschaft zu kreieren.

Durch dieses Konsumverhalten zusammen mit der stärkeren Konkurrenz an diesem transparenteren Markt, sowie den technischen Limitationen des Mediums, stellt sich das Monetarisieren von digitalen Inhalten als eine nicht zu unterschätzende Herausforderung dar.

Das einzige Geschäftsmodell, welches sich bisher wirklich durchgesetzt hat ist die Monetarisierung durch das Schalten von Werbeanzeigen. Dieses bringt allerdings einige Problemen mit sich. Durch dieses Modell wird die Wirtschaftlichkeit eines Inhalts ausschließlich durch die von ihm generierte Anzahl an Aufrufen bestimmt. Dadurch wird der Konsument der Inhalte vom Kunden der Publikationen zu deren Produkt, welches diese an die Werbeschaltenden verkaufen. Dadurch rückt der eigentliche Kunde aus dem Fokus und die Tatsache dass Werbeanzeigen generell als störend empfunden werden, wird hingenommen. Tendenziell belohnt das Modell Quantität mehr als Qualität und dadurch kurze, Aufmerksamkeit heischende Beiträge mehr als fundierte, differenzierte Berichterstattung. Vor allem aber, rechnet es sich erst ab einer hohen absoluten Zahl von Aufrufen und ist auch dann oft nicht ausreichend. Zum Beispiel erwirtschaftet die international erfolgreiche, britische Publikation \enquote{The Guardian} seit Jahren nur Verluste und muss sich durch andere Publikationen querfinanzieren.

Hieraus ergibt sich eine Situation, welche sowohl für den Konsumenten als auch für den Produzenten der Inhalte unvorteilhaft ist und wir wurden das Gefühl nicht los, dass es doch eine bessere Lösung geben muss.

\subsection{Konzept}

Wir wollten eine Lösung gestalten, die dem Nutzer all die Freiheit und Flexibilität gibt, die er von einem modernen Dienst gewohnt ist und erwartet, jedoch auch für die Anbieter von Inhalten eine effektive und faire Form der Monetarisierung darstellt. Das ganze sollte abgerundet werden durch eine intuitive und leichte Nutzererfahrung, welche die Hürden für das Verwenden unseres Dienstes und das Konsumieren von monetarisierten Inhalten durch ihn so weit wie möglich senkt.

Hierfür haben wir eine Nachrichten App entworfen, welche Inhalte von Verschiedensten Quellen auf einer möglichst granularen Ebene bereitstellt. Das bedeutet, dass man grundsätzlich für jeden Inhalt einzeln bezahlen kann oder aber auch bestimmte Sammlungen von Inhalten abonnieren kann. In der Praxis könnte man so das \enquote{Sport}-Segment von Zeitung A abonnieren, sowie alle Artikel die von Autor P geschrieben werden, die Komplette Zeitung B, als auch vereinzelte Artikel von den Zeitschriften C, D und E. Wir versuchen so die Vorteile von On-Demand- sowie Flatrate-Modellen zu nutzen. Ein Nutzer kann sich die von ihr gekauften oder abonnierten Inhalte dann in sogenannten Feeds zusammenstellen und so ihren Konsum stark personalisieren.

Bisherige Produkte, welche ehemals Print-Inhalte digital anbieten, haben zwar einige der beschriebenen Probleme gelöst indem sie ihren Kunden einen Single-Point-of-Access (SPA) für die Inhalte mehrerer Publikationen sowie die damit zusammenhängenden Transaktionen bieten. Allerdings, haben sie nicht den revolutionären Schritt getan die Struktur der Angebotenen Inhalte aufzubrechen und den Nutzern somit umfangreiche Personalisierungsmöglichkeiten beim Konsum als auch Flexibilität beim Kauf zu ermöglichen. Deswegen sind wir überzeugt, dass sich unser Produkt in dem Oligopol der digitalen Nachrichten-Plattformen gegen die Konkurrenz behaupten kann.

Während wir in dieser Ausarbeitung nur die Umsetzung unseres Konzepts als Mobile-App ausführen, so wären in einem realen Szenario jedoch auch eine Webapp, sowie angepasste Anwendungen für alle anderen gängigen Systeme Teil der Umsetzung. Hierbei würde man sich um ein möglichst eingängiges Design bemühen um Nutzern ein möglichst uniformes Erlebnis zu ermöglichen, sowie von Vertrautheit und Wiedererkennungswert zu profitieren.
